\documentclass{article}

%%%%%%%%%%%%%%%%%%%%%%%%%%%%%%%%%%% PACKAGES %%%%%%%%%%%%%%%%%%%%%%%%%%%%%%%%%%

\usepackage{arxiv}

\usepackage[utf8]{inputenc} % allow utf-8 input
\usepackage[T1]{fontenc}    % use 8-bit T1 fonts
\usepackage{url}            % simple URL typesetting
\usepackage{booktabs}       % professional-quality tables
\usepackage{amsfonts}       % blackboard math symbols
\usepackage{nicefrac}       % compact symbols for 1/2, etc.
\usepackage{microtype}      % microtypography
\usepackage{lipsum}
\usepackage{verbatim}       % Multiline comments

%% Mathematics
\RequirePackage{amssymb}    % Extra symbols
\RequirePackage{amsthm}     % Theorem-like environments
\RequirePackage{thmtools}   % Theorem-like environments, extends amsthm
\RequirePackage{mathtools}  % Fonts and environments for mathematical formulae
\RequirePackage{mathrsfs}   % Script font with \mathscr{}
\RequirePackage{cancel}     % Cancel terms with \cancel, \bcancel or \xcancel
\RequirePackage{stmaryrd}   % Brackets

%% Drawing
\RequirePackage{tikz}             % Drawing tool
\usetikzlibrary{calc}
\usetikzlibrary{intersections}
\usetikzlibrary{decorations.markings}
\usetikzlibrary{cd} % An implementation of simple commutative diagrams in tikz
\RequirePackage[all]{xy} % For backwards compatibility with respect to publishing articles.


%% Cross references and links
\RequirePackage{varioref}                                     % \vref
\RequirePackage{hyperref}                                     % Clickable links
\RequirePackage[nameinlink, capitalize, noabbrev]{cleveref}   % \cref
\RequirePackage{doi}           % Ignore LaTeX syntax in DOI links
\renewcommand{\doitext}{DOI:}
\urlstyle{sf}


%%%%%%%%%%%%%%%%%%%%%%%%%%%%% USER-DEFINED MACROS %%%%%%%%%%%%%%%%%%%%%%%%%%%%%

%% Environments
\declaretheoremstyle[headfont   = \bfseries\sffamily,
                     notefont   = \normalfont,
                     spaceabove = 6pt plus 0pt minus 2pt]{plain}
\declaretheoremstyle[headfont   = \bfseries\sffamily,
                     notefont   = \normalfont,
                     spaceabove = 6pt plus 0pt minus 2pt]{definition}
\declaretheorem[style = plain, numberwithin = section]{theorem}
\declaretheorem[style = plain,      sibling = theorem]{corollary}
\declaretheorem[style = plain,      sibling = theorem]{lemma}
\declaretheorem[style = plain,      sibling = theorem]{proposition}
\declaretheorem[style = plain,      sibling = theorem]{observation}
\declaretheorem[style = plain,      sibling = theorem]{conjecture}
\declaretheorem[style = definition, sibling = theorem]{definition}
\declaretheorem[style = definition, sibling = theorem]{example}
\declaretheorem[style = definition, sibling = theorem]{notation}
\declaretheorem[style = remark,     sibling = theorem]{remark}
\declaretheorem[style = definition, numbered = no]{acknowledgements}
\crefname{observation}{Observation}{Observations}
\Crefname{observation}{Observation}{Observations}
\crefname{conjecture}{Conjecture}{Conjectures}
\Crefname{conjecture}{Conjecture}{Conjectures}
\crefname{notation}{Notation}{Notations}
\Crefname{notation}{Notation}{Notations}
\crefname{paper}{Paper}{Papers}
\Crefname{paper}{Paper}{Papers}


%% Operators
\DeclareMathOperator{\im}{im}


%% Delimiters
\DeclarePairedDelimiter{\p}{\lparen}{\rparen}          % Parenthesis
\DeclarePairedDelimiter{\set}{\lbrace}{\rbrace}        % Set
\DeclarePairedDelimiter{\abs}{\lvert}{\rvert}          % Absolute value
\DeclarePairedDelimiter{\norm}{\lVert}{\rVert}         % Norm
\DeclarePairedDelimiter{\ip}{\langle}{\rangle}         % Inner product, ideal
\DeclarePairedDelimiter{\sqb}{\lbrack}{\rbrack}        % Square brackets
\DeclarePairedDelimiter{\ssqb}{\llbracket}{\rrbracket} % Double brackets
\DeclarePairedDelimiter{\ceil}{\lceil}{\rceil}         % Ceiling
\DeclarePairedDelimiter{\floor}{\lfloor}{\rfloor}      % Floor


%% Sets
\newcommand{\N}{\mathbb{N}}    % Natural numbers
\newcommand{\Z}{\mathbb{Z}}    % Integers
\newcommand{\Q}{\mathbb{Q}}    % Rational numbers
\newcommand{\R}{\mathbb{R}}    % Real numbers
\newcommand{\C}{\mathbb{C}}    % Complex numbers
\newcommand{\A}{\mathbb{A}}    % Affine space
\renewcommand{\P}{\mathbb{P}}  % Projective space


%% Vectors
\renewcommand{\a}{\mathbf{a}}
\renewcommand{\b}{\mathbf{b}}
\renewcommand{\c}{\mathbf{c}}
\newcommand{\x}{\mathbf{x}}
\newcommand{\y}{\mathbf{y}}
\newcommand{\0}{\mathbf{0}}

%% Miscellaneous
\renewcommand{\qedsymbol}{\(\blacksquare\)}
\newcommand{\ie}{\leavevmode\unskip, i.e.,\xspace}
\newcommand{\eg}{\leavevmode\unskip, e.g.,\xspace}
\newcommand{\dash}{\textthreequartersemdash\xspace}
\newcommand{\TikZ}{Ti\textit{k}Z\xspace}
\newcommand{\matlab}{\textsc{Matlab}\xspace}



\title{Exponentials of Non-singular Simplicial Sets}


\author{
  Vegard Fjellbo \\
  Department of Mathematics \\
  University of Oslo \\
  Oslo, Norway \\
  \texttt{vegard.fjellbo@gmail.com} \\
  %% examples of more authors
   \And
  John Rognes \\
  Department of Mathematics \\
  University of Oslo \\
  Oslo, Norway \\
  \texttt{rognes@math.uio.no} \\
  %% \AND
  %% Coauthor \\
  %% Affiliation \\
  %% Address \\
  %% \texttt{email} \\
  %% \And
  %% Coauthor \\
  %% Affiliation \\
  %% Address \\
  %% \texttt{email} \\
  %% \And
  %% Coauthor \\
  %% Affiliation \\
  %% Address \\
  %% \texttt{email} \\
}

\setlength{\parindent}{1em}


\begin{document}
\maketitle


\begin{abstract}
\noindent A simplicial set is \textbf{non-singular} if the representing maps of its non-degenerate simplices are degreewise injective. The category of simplicial sets has a \textbf{simplicial mapping set} $X^K$ whose set of $n$-simplices are the simplicial maps $\Delta [n]\times K\to X$. We prove that $X^K$ is non-singular whenever $X$ is non-singular.

\vspace{1em}
MSC-class: 18D15 (Primary), 55U10 (Secondary)
\end{abstract}



% keywords can be removed
\keywords{Cartesian Closed \and Non-singular \and Simplicial sets}

\section{Introduction}
\label{sec:intro_exp}

There are times when one would like to know whether a category behaves similarly, in some sense, to the category of sets and functions. As an example, for homotopy-theoretical purpose the author would like to know whether the endofunctor $-\times \Delta [1]$ of non-singular simplicial sets preserves colimits. Here, $\Delta [1]$ denotes the standard $1$-simplex.

Let $sSet$ denote the category of simplicial sets. The full subcategory $nsSet$ whose objects are the non-singular simplicial sets sits strictly between $sSet$ and the category of ordered simplicial complexes. Despite the fact that non-singular simplicial sets have a natural PL structure \cite[p.~126--127]{WJR13} they almost never appear in the literature, though they do play a role in the book Spaces of PL Manifolds and Categories of Simple Maps by Waldhausen, Jahren and Rognes \cite{WJR13}.

The endofunctor $(-)^K:sSet\to sSet$ is designed so that the Yoneda lemma makes it right adjoint to $-\times K$. Our main result is the following.
\begin{theorem}\label{thm:arbitrary_exponent}
Let $K$ be some simplicial set. Then $X^K$ is non-singular whenever $X$ is.
\end{theorem}
\noindent Part of the author's interest in this result comes from the case when $K$ non-singular. Then the restriction of $(-)^K$ to $nsSet$ corestricts to an endofunctor of non-singular simplicial sets. Moreover, $(-)^K$ viewed as a functor $nsSet\to nsSet$ is right adjoint to the endofunctor $-\times K$ of $nsSet$. This means that we can derive the following consequence of \cref{thm:arbitrary_exponent}.
\begin{corollary}\label{cor:take_product_cocontinous_endofunctor_non-singular}
Taking the product $-\times K:nsSet\to nsSet$ with a non-singular simplicial set $K$ preserves colimits.
\end{corollary}
\noindent In particular, taking the product $-\times \Delta [1]$ with an interval is a cocontinous endofunctor of non-singular simplicial sets.

The case of the interval is not only of practicle concern, but it is also the theoretical focus of this article as it is not hard to argue that \cref{thm:arbitrary_exponent} follows from the following result.
\begin{proposition}\label{prop:standard_1_simplex_as_exponent}
The simplicial set $X^{\Delta [1]}$ is non-singular whenever $X$ is.
\end{proposition}
\noindent The proof of the latter result is the subject of \cref{sec:prism}, whereas \cref{thm:arbitrary_exponent} is derived from \cref{prop:standard_1_simplex_as_exponent} in \cref{sec:arbexp}.

In \cref{sec:app}, we will discuss applications of \cref{thm:arbitrary_exponent} beyond \cref{cor:take_product_cocontinous_endofunctor_non-singular}. We explain how \cref{thm:arbitrary_exponent} follows from \cref{prop:standard_1_simplex_as_exponent} in \cref{sec:arbexp}. Finally, the case of the interval is discussed \cref{sec:prism}.






\section{Applications}
\label{sec:app}

The inclusion $U:nsSet\to sSet$ admits a left adjoint functor called desingularization \cite[Rem.~2.2.12.,~p.~39]{WJR13}, which is denoted $D$. Note that the unit
\[\eta _X:X\to UDX\]
is degreewise surjective and that desingularization has the universal property that any simplicial map $f:X\to Y$ whose target $Y$ is non-singular factors through the unit by a unique map $UDX\to Y$.

In general, we say that a full subcategory of some category is a \textbf{reflective subcategory} if the inclusion admits a left adjoint, which is then referred to as a \textbf{reflector}. Thus $nsSet$ is a reflective subcategory of $sSet$. Note that the word reflective is not quite standard terminology. For example, Mac Lane \cite[§IV.3]{ML98} Adámek and Rosický \cite[p.~1306]{AR15} do not include fullness as an assumption in their definition, although some other authors do. \cref{prop:standard_1_simplex_as_exponent} and its generalization \cref{thm:arbitrary_exponent} has a noteworthy application and a couple of consequences.

The main theorem of \cite{Fj20-HTY} establishes a model structure on $nsSet$ that is right-induced a la Thomason \cite{Th80} from $sSet$ equipped with the standard model structure due to Quillen \cite{Qu67}. Moreover, the theorem says that $(D,U)$ is a Quillen equivalence. \cref{prop:standard_1_simplex_as_exponent} is used as a technical ingredient in the proof of that theor.

Another way to state \cref{thm:arbitrary_exponent} is to say that the non-singular simplicial sets form an exponential ideal in $sSet$. The category of simplicial sets is cartesian closed and even a topos. Part of this is the fact that $(-)^K$ is right adjoint to $-\times K$. Here, the author has in mind the notions, terminology and notation from \cite[§IV.6--§IV.10]{ML98}. Note that the construction $X^K$ is bifunctorial. A generalized result known as the parameter theorem ensures this \cite[p.~102]{ML98}.
\begin{corollary}\label{cor:desingularization_preserves_finite_products_cartesian_closed}
Desingularization preserves finite products.
\end{corollary}
\noindent It seems that \cref{cor:desingularization_preserves_finite_products_cartesian_closed} follows from Day's reflection theorem \cite[Thm.~1.2]{Da72} and its corollary \cite[Cor.~2.1]{Da72}. Day's reflection theorem concerns a more general setting, although he does refer to the condition that the \emph{reflective subcategory is closed under exponentiation} \cite[§0]{Da72}. Another phrase that is used in the literature is that the non-singular simplicial sets form an \emph{exponential ideal} in $sSet$, which is exactly the content of \cref{thm:arbitrary_exponent}.

In case one does not want to unravel the general form of Day's reflection theorem, we provide the following elementary proof.
\begin{proof}[Proof of \cref{cor:desingularization_preserves_finite_products_cartesian_closed}.]
It is enough to consider two factors. Suppose $X$ and $Y$ simplicial sets.

Consider the map
\[Y\times X\xrightarrow{\eta _{Y\times X}} D(Y\times X).\]
Here, we omit the redundant symbol $U$ for the inclusion functor. By \cref{thm:arbitrary_exponent}, the simplicial set $D(Y\times X)^X$ is non-singular, so we obtain a factorization
\begin{equation}
\label{eq:first_diagram_proof_cor_desingularization_preserves_finite_products_cartesian_closed}
\begin{gathered}
\xymatrix{
Y \ar[dr] \ar[rr]^{\eta _Y} && DY \ar@{-->}[ld] \\
& D(Y\times X)^X
}
\end{gathered}
\end{equation}
of the adjoint. Next, we switch the two factors of the adjoint
\[DY\times X\to D(Y\times X)\]
of the dashed map in (\ref{eq:first_diagram_proof_cor_desingularization_preserves_finite_products_cartesian_closed}) and factor the adjoint of the resulting map by means of the diagram
\begin{equation}
\label{eq:second_diagram_proof_cor_desingularization_preserves_finite_products_cartesian_closed}
\begin{gathered}
\xymatrix{
X \ar[dr] \ar[rr]^{\eta _X} && DX \ar@{-->}[ld] \\
& D(X\times Y)^{DY}
}
\end{gathered}
\end{equation}
in which the dashed map arises by \cref{thm:arbitrary_exponent} as $D(X\times Y)^{DY}$ is non-singular.

By adjunction, we can combine (\ref{eq:first_diagram_proof_cor_desingularization_preserves_finite_products_cartesian_closed}) and (\ref{eq:second_diagram_proof_cor_desingularization_preserves_finite_products_cartesian_closed}) into the solid commutative diagram
\begin{equation}
\label{eq:third_diagram_proof_cor_desingularization_preserves_finite_products_cartesian_closed}
\begin{gathered}
\xymatrix{
X\times Y \ar[dr]_{\eta _{Y\times X}} \ar[rr]^{id\times \eta _X} && X\times DY \ar[ld] \ar[dr]^{\eta _X\times id} \\
& D(X\times Y) \ar@{-->}@/_1pc/[rr]_{(D(pr_1),D(pr_2))} && DX\times DY \ar[ll]
}
\end{gathered}
\end{equation}
in which a dashed map arises because $DX\times DY$ is non-singular, being a product of non-singular simplicial sets. Indeed, the dashed map must be equal to the canonical map $(D(pr_1),D(pr_2))$ due to the universal property of desingularization.

Because the map $\eta _{X\times Y}$ is degreewise surjective and because (\ref{eq:third_diagram_proof_cor_desingularization_preserves_finite_products_cartesian_closed}) commutes, it follows immediately that
\[DX\times DY\to D(X\times Y)\]
is degreewise surjective.

Furthermore, by the universal property of desingularization, it follows that the composite
\[DX\times DY\to D(X\times Y)\xrightarrow{(D(pr_1),D(pr_2))} DX\times DY\]
is the identity. This implies that the first of the two maps of the composite is even degreewise injective, which implies that it is degreewise bijective and hence an isomorphism. In this way, we see that $(D(pr_1),D(pr_2))$ is degreewise bijective and hence an isomorphism.
\end{proof}
\noindent Another consequence of \cref{thm:arbitrary_exponent} is the following result.
\begin{corollary}
The category of non-singular simplicial sets is cartesian closed.
\end{corollary}

% Is $nsSet$ a topos in the sense of Mac Lane? I suppose I have to study the subobject classifier of $sSet$ to be able to answer this question. According to

% https://www.encyclopediaofmath.org/index.php/Reflective_subcategory

% a morphism in a (full) reflective subcategory is a monomorphism if and only if it is so in the surrounding category. And limits are formed in $nsSet$ as they are formed in $sSet$. Is $\Omega$, the subobject classifier of $sSet$, a non-singular simplicial set? If not, what is its desingularization? What is gained by being able to answer this question?

% Subobject classifiers in the sense of Mac Lane are discussed here:

% https://mathoverflow.net/questions/159989/internal-logic-of-the-topos-of-simplicial-sets
% https://en.wikipedia.org/wiki/Subobject_classifier
% https://ncatlab.org/nlab/show/subobject+classifier
% https://ncatlab.org/toddtrimble/published/Monic+endomorphisms+on+the+subobject+classifier






\section{Arbitrary exponent}
\label{sec:arbexp}

In this section we will prove \cref{thm:arbitrary_exponent}, assuming that \cref{prop:standard_1_simplex_as_exponent} holds. First we will point out that the latter result can be generalized fairly easily from the interval to the standard $n$-simplex, for all $n\geq 0$.
\begin{lemma}\label{lem:standard_n_simplex_as_exponent}
Suppose $n\geq 0$. The simplicial set $X^{\Delta [n]}$ is non-singular if $X$ is.
\end{lemma}
\noindent To verify \cref{lem:standard_n_simplex_as_exponent} we note that \cref{prop:standard_1_simplex_as_exponent} implies that $X^{\Delta [1]^n}$ is non-singular if $X$ is. This is by induction on $n$, which is made possible by the exponential law $(X^K)^L\cong X^{L\times K}$, which holds because $sSet$ is cartesian closed.

Let $[n]$ denote the totally ordered set $\{ 0<1<\cdots <n\}$. Following \cite[p.~132]{FP90}, we shall refer to an \textbf{operator} as a function $\alpha :[m]\to [n]$ such that $\alpha (i)\leq \alpha (j)$ if $i\leq j$. Observe that $\Delta [n]$ embeds in $\Delta [1]^n$ in such a way that $\Delta [1]^n$ retracts onto $\Delta [n]$. The embedding $i$ that we have in mind is induced by the operator
\[[n]\to [1]^n\]
given by
\[j\mapsto 1\dots 10\dots 0\]
where the string $1\dots 10\dots 0$ starts with $j$ $1$'s and the rest are $0$'s. One can make a retraction $r:\Delta [1]^n\to \Delta [n]$ by taking the string $k_1\dots k_n$ from $[1]^n$ and then finding the lowest index $j$ such that $k_j=0$. Then one defines an operator by the rule
\[k_1\dots k_n\mapsto j-1,\]
which induces the announced $r$. We get that the composite $ri$ is the identity as this is true on the level of operators.

There are induced maps
\[X^{\Delta [n]}\xleftarrow{X^i} X^{\Delta [1]^n}\xleftarrow{X^r} X^{\Delta [n]}\]
such that the composite is equal to the identity. In other words, the simplicial set $X^{\Delta [n]}$ is identified with a simplicial subset of $X^{\Delta [1]^n}$, which is non-singular if $X$ is. Hence, the simplicial set $X^{\Delta [n]}$ is non-singular if $X$ is. This concludes our proof of \cref{lem:standard_n_simplex_as_exponent}, given that \cref{prop:standard_1_simplex_as_exponent} holds.

By means of \cref{lem:standard_n_simplex_as_exponent}, we can derive our main result.
\begin{proof}[Proof of \cref{thm:arbitrary_exponent}.]
Suppose $K$ is some simplicial set and let $X$ be non-singular. Let $\Delta \downarrow K$ denote the \textbf{simplex category}, meaning the category whose objects are the pairs $(x,n)$, where $x$ is a simplex of $K$ whose degree is $n$, and whose morphisms $(y,m)\to (x,n)$ are the pairs $(x,\alpha )$ with $\alpha$ an operator such that $y=x\alpha$.

The simplicial set $K$ can be viewed as the colimit of the diagram
\[\Upsilon _K:\Delta \downarrow K\to sSet\]
that sends a simplex of degree $n$ to the standard $n$-simplex $\Delta [n]$\\ \cite[Lem.~4.2.1]{FP90}. We explain that $X^K$ is the limit of the composite
\[\Delta \downarrow K\xRightarrow{\Upsilon _K} sSet\xRightarrow{X^{(-)} } sSet,\]
denoted $X^{\Upsilon _K}$, or in other words that the cone $\underline{X^K} \Rightarrow X^{\Upsilon _K}$ is universal.

Assume that $\underline{Z} \Rightarrow X^{\Upsilon _K}$ is a cone. Recall that $sSet$ is cartesian closed. Via the natural bijection
\[sSet(Z\times \Delta [n],X)\xrightarrow{\cong } sSet(Z,X^{\Delta [n]}),\]
we can consider the cocone $Z\times X^{\Upsilon _K}\Rightarrow \underline{X}$ illustrated in the diagram
\begin{displaymath}
\xymatrix{
Z\times \Delta [m] \ar[dd]_{id\times \alpha } \ar[dr]_{id\times \bar{y} } \ar@/^2pc/[drr] \\
& Z\times K \ar@{-->}[r]^{\exists !} & X \\
Z\times \Delta [n] \ar[ur]^{id\times \bar{x} } \ar@/_2pc/[urr]
}
\end{displaymath}
instead. Because $Z\times -$ is a cocontinous endofunctor of simplicial sets, the simplicial set $Z\times K$ is the colimit of $Z\times \Upsilon _K$. Hence, there exists a (unique) map $Z\times K\to X$ that gives rise to a factorization of the cocone $Z\times X^{\Upsilon _K}\Rightarrow \underline{X}$. By adjointness, we obtain a map $Z\to X^K$ such that the given, arbitrary cone on $X^{\Upsilon _K}$ factors through $\underline{X^K} \Rightarrow X^{\Upsilon _K}$.

On the other hand, any map $Z\to X^K$ that gives rise to such a factorization corresponds to a map $Z\times K\to X$ that factors the cocone $Z\times \Upsilon _K\Rightarrow \underline{X}$ through the universal cocone. However, there is only one map $Z\times K\to X$ of the latter type. By adjointness, the map $Z\to X^K$ is therefore unique.

The diagram $X^{\Upsilon _K}$ is by \cref{lem:standard_n_simplex_as_exponent} a diagram whose objects are non-singular. Because $nsSet$ is a reflective subcategory of $sSet$, it follows that $X^K$ is non-singular \cite[p.~1306]{AR15}.
\end{proof}
\noindent In the proof of \cref{thm:arbitrary_exponent}, we used the non-trivial fact that a reflective subcategory inherits limits from its surrounding category, although we could have argued in more elementary terms.

According to Adámek and Rosický \cite[p.~1306]{AR15}, the earliest proof that appears in the literature, of the inheritance of limits by reflective subcategories, is to be found in the works of H. Herrlich \cite{He68}.




\section{Rigidity of the prism}
\label{sec:prism}

We give a proof that $X^{\Delta [1]}$ is non-singular whenever $X$ is non-singular. This is the claim presented in \cref{prop:standard_1_simplex_as_exponent}. An informal way of stating this result is to say that prisms on non-singular simplicial sets are very rigid. Recall that \cref{sec:arbexp} explains how to derive \cref{thm:arbitrary_exponent} from \cref{prop:standard_1_simplex_as_exponent}. Thus the work of this section finishes the proof of our main result.

For convenience, we introduce some terminology and notation before we present the proof. An injective operator is said to be a \textbf{face operator} and a surjective operator is said to be a \textbf{degeneracy operator}. Special face operators are the \textbf{elementary face operators} $\delta ^n_i:[n-1]\to [n]$ that omit the index $i$ and \textbf{vertex operators} $\varepsilon ^n_i:[0]\to [n]$ that hit the indices $i$. Special degeneracy operators are the \textbf{elementary degeneracy operators} $\sigma ^n_i:[n+1]\to [n]$ that send $i$ and its successor $i+1$ to $i$. Frequently, we omit the upper index in the notation. Similar to the terminology in \cite{WJR13}, we will refer to $\delta ^n_n\dots \delta ^q_q:[q-1]\to [n]$, $0<q\leq n$, as the \textbf{$q$-th front face} of $[n]$ and to $\delta ^n_p\dots \delta ^{n-p}_0:[n-(p+1)]\to [n]$, $0\leq p<n$, as the \textbf{$p$-th back face} of $[n]$.

A face operator or degeneracy operator is \textbf{proper} if it is not the identity. Consider a simplicial set. A simplex $y$ is a \textbf{(proper) face} of another simplex $x$ if $y=x\mu$ for a (proper) face operator $\mu$. Analogously, a simplex $y$ is a \textbf{(proper) degeneracy} of another simplex $x$ if $y=x\rho$ for a (proper) degeneracy operator $\rho$. A simplex is \textbf{degenerate} if it is a proper degeneracy of some simplex. Otherwise, it is said to be \textbf{non-degenerate}.

In the proof, we will use the Eilenberg-Zilber lemma \cite[Thm.~4.2.3]{FP90}, which says that any simplex $x$ of any simplicial set $X$ is uniquely a degeneration $x=x^\sharp x^\flat$ of some non-degenerate simplex $x^\sharp$. We say that $x^\sharp$ is the \textbf{non-degenerate part} of $x$, following \cite{WJR13}, and that $x^\flat$ is the \textbf{degenerate part} of $x$. Note that $x$ and $x^\sharp$ are objects in the category $\Delta \downarrow X$ while $x^\flat$ can be regarded as a morphism $x\to x^\sharp$. Thus the terminology is not perfect, however it is useful. According to the Yoneda lemma, the $n$-simplices $x$ of a simplicial set $X$ are in natural bijective correspondence $x\mapsto \bar{x}$ with the simplicial maps $\Delta [n]\to X$. The map $\bar{x}$ is the \textbf{representing map} of $x$. We say that a simplex is \textbf{embedded} if its representing map is degreewise injective.

Because of the new terminology, we get a shorter definition of \emph{non-singular} in the second condition of \cref{lem:non-singular_equivalent_criteria}, below. Furthermore, there is another formulation that is useful in the proof of \cref{prop:standard_1_simplex_as_exponent}, though a bit awkward. It is given as the third condition \cref{lem:non-singular_equivalent_criteria}
\begin{lemma}\label{lem:non-singular_equivalent_criteria}
The following statements are equivalent.
\begin{enumerate}
\item{The simplicial set $X$ is non-singular.}
\item{Each non-degenerate simplex of $X$ is embedded.}
\item{Eeach simplex of $X$ is degenerate provided its vertices are not pairwise distinct.}
\end{enumerate}
\end{lemma}
\noindent The equivalence of the second and third statement is somewhat refined by the next lemma.
\begin{lemma}\label{lem:degenerate_part_factorization_through}
Let $X$ be a non-singular simplicial set and $x$ some simplex with $z\varepsilon _k=z\varepsilon _l$. Then the degenerate part $x^\flat$ of $x$ factors uniquely through the degeneracy operator $\sigma _k\dots \sigma _{l-1}$.
\end{lemma}
\begin{proof}
Write $\rho =\sigma _k\dots \sigma _{l-1}$. The uniqueness of a factorization of $x^\flat$ through $\rho$ is automatic as $\rho$ is epic in $Cat$. It is the existence part that requires an argument.

Because $X$ is non-singular it follows that the non-degenerate part $x^\sharp$ is embedded, which is the same as saying that its vertices are pairwise distinct. This means that $x^\flat (k)=x^\flat (l)$. As $x^\flat$ is order-preserving, it follows that $x^\flat (j)=x^\flat (k)$ if $k\leq j\leq l$. Thus $\rho (i)=\rho (j)$ implies $x^\flat (i)=x^\flat (j)$. Take a section $\mu$ of $\rho$. We get that $x^\flat =(x^\flat \mu )\rho$.
\end{proof}
\noindent \cref{lem:degenerate_part_factorization_through} will be used to break down the proof of \cref{prop:standard_1_simplex_as_exponent} into two parts.

If $x$ is some simplex, say of degree $n$, whose degenerate part factors through the elementary degeneracy operator $\sigma _k$ for some $k$ with $0\leq k<n$, then we will say that $x$ \textbf{splits off} $\sigma _k$. In particular, if $X$ is non-singular and if $x$ is a simplex of $X$ such that $x\varepsilon _k=x\varepsilon _{k+1}$, then $x$ splits off $\sigma _k$ according to \cref{lem:degenerate_part_factorization_through}.

The canonical identification
\[N([n]\times [1])\xrightarrow{\cong } \Delta [n]\times \Delta [1]\]
gives us a preferred set of generators of the prism $\Delta [n]\times \Delta [1]$, namely the $n+1$ non-degenerate
$(n+1)$-simplices
\[\gamma ^{n+1}_j:[n+1]\rightarrow [n]\times [1],\]
$0\leq j\leq n$,
given by
\begin{displaymath}
\gamma ^{n+1}_j(i)=
\begin{cases}
(i,0), & 0 \leq i \leq j \\ 
(i-1,1), & j < i \leq n.
\end{cases}
\end{displaymath}
Coming from the diagram
\begin{displaymath}
\xymatrix{
& \dots \ar[r] & (j,1) \ar[r] & (j+1,1) \ar[r] & \dots \ar[r] & (n,1) \\
(0,0) \ar[r] & \dots \ar[r] & (j,0) \ar[u] \ar[r] \ar[ur] & (j+1,0) \ar[r] \ar[u] & \dots
}
\end{displaymath}
are the conditions
\begin{equation}\label{Equation_generator_compatibility}
\gamma ^{n+1}_j\delta _{j+1}=\gamma ^{n+1}_{j+1}\delta _{j+1}
\end{equation}
for $0\leq j\leq n$. These conditions, which can be thought of glueing conditions for constructing the prism from $n+1$
copies of the standard $(n+1)$-simplex, generate all relations that the generators satisfy.

We are done with the setup and are ready to prove \cref{prop:standard_1_simplex_as_exponent}. Suppose $X$ non-singular. Keep in mind the third and equivalent way to state this, as formulated in \cref{lem:non-singular_equivalent_criteria}. The proof is divided into two parts, the first of which is the following result.
\begin{lemma}\label{lem:part1_standard_1_simplex_as_exponent}
Assume that $\Phi$ is an $n$-simplex of $X^{\Delta [1]}$ such that the $k$-th vertex and the $l$-th vertex are equal, for some $k$ and some $l$ with $0\leq k<l\leq n$. Then
\[\Phi \varepsilon _k=\Phi \varepsilon _{k+1}=\dots =\Phi \varepsilon _l.\]
\end{lemma}
\noindent The second part is \cref{lem:part2_standard_1_simplex_as_exponent}, where we prove that any given $n$-simplex $\Phi$ of $X^{\Delta [1]}$ is degenerate if it is such that the $k$-th vertex is equal to the $(k+1)$-th vertex, for some $k$ with $0\leq k<n$.

Thus, by \cref{lem:part1_standard_1_simplex_as_exponent} and \cref{lem:part2_standard_1_simplex_as_exponent}, any simplex of $X^{\Delta [1]}$ is degenerate provided its vertices are not pairwise distinct. Lemma \cref{lem:non-singular_equivalent_criteria} then says that $X^{\Delta [1]}$ is non-singular. We can therefore conclude that \cref{prop:standard_1_simplex_as_exponent} holds when we have proven the two lemmas.

\begin{proof}[Proof of \cref{lem:part1_standard_1_simplex_as_exponent}.]
Suppose $\Phi$ an $n$-simplex of $X^{\Delta [1]}$ such that $\Phi\varepsilon _k=\Phi \varepsilon _l$ for some $k$ and some $l$ with $0\leq k<l\leq n$. What is immediately noticeable is that the composite of $\Phi$ with the inclusion of the bottom of the prism is an $n$-simplex
\[x_0=\Phi \circ (id,N\varepsilon _0)\]
of $X$ whose $k$-th and $l$-th vertex are also equal. Doing something similar at the top of the prism we get a simplex
$x_1=\Phi \circ (id,N\varepsilon _1)$ with $x_1\varepsilon _k=x_1\varepsilon _l$.

From \cref{lem:degenerate_part_factorization_through} it follows that the degenerate part $x_0^\flat$ of $x_0$ factors uniquely through $\sigma _k\dots \sigma _{l-1}$. Thus we can write
\begin{displaymath}
\begin{array}{rcl}
x_0 & = & y_0\sigma _k\dots \sigma _{l-1} \\
x_1 & = & y_1\sigma _k\dots \sigma _{l-1}
\end{array}
\end{displaymath}
for some $(k+n-l)$-simplices $y_0$ and $y_1$ of $X$.

Suppose $k\leq j<l$. Writing $x_0$ and $x_1$ as degenerations indicates that the $(n+1)$-simplices $\Phi (\gamma ^{n+1}_{j+1})$ and $\Phi (\gamma ^{n+1}_j)$ of $X$ must be degenerate. To answer how they are degenerate, form the left hand cartesian square in the following diagram.
\begin{displaymath}
\xymatrix{
\Delta [n+1] \ar[r]^(.45){\gamma ^{n+1}_{j+1}} & \Delta [n]\times \Delta [1] \ar[r]^(.6)\Phi & X \\
\Delta [j+1] \ar[u] \ar[r] & \Delta [n] \ar[u]^{(id,N\varepsilon _0)} \ar[ur]_{x_0} \ar[r]_(.4){N(\sigma _k\dots \sigma _{l-1})} & \Delta [k+n-l] \ar[u]_{y_0}
}
\end{displaymath}
The canonical map $\Delta [j+1]\to \Delta [n+1]$ is then induced by the $(j+2)$-th front face of $[n+1]$ and the canonical map $\Delta [j+1]\to \Delta [n]$ is induced by the $(j+2)$-th front face of $[n]$.

The above implies that the $j$-th and the $(j+1)$-th vertex of $\Phi (\gamma ^{n+1}_{j+1})$ are equal. A similarly constructed diagram involving $x_1$, $y_1$, $(id,N\varepsilon _1)$ and $\Phi (\gamma ^{n+1}_j)$ shows that the $(j+1)$-th and the $(j+2)$-th vertex of $\Phi (\gamma ^n_j)$ are equal.

As a consequence of the previous paragraph, we will argue that the $j$-th and the $(j+1)$-th vertex of the $n$-simplex $\Phi$ of $X^{\Delta [1]}$ are equal. They are the vertices of the $1$-simplex
\[\Delta [1]\times \Delta [1]\xrightarrow{N\mu \times id} \Delta [n]\times \Delta [1]\xrightarrow{\Phi } X,\]
of $X^{\Delta [1]}$ where $\mu$ is given by $0\mapsto j$ and $1\mapsto j+1$.

We can view the vertices $\Phi \varepsilon _j$ and $\Phi \varepsilon _{j+1}$ of the simplex $\Phi$ of $X^{\Delta [1]}$ as
$1$-simplices of $X$. When we do, they fit into the commutative diagram
\begin{displaymath}
 \xymatrix{
 & \Delta [2] \ar[dr]^{\gamma ^2_1} && \Delta [1] \ar[ll]_{\delta _0} \ar[dr]^{\Phi \varepsilon _{j+1}} \\
 \Delta [1] \ar[ur]^{\delta _1} \ar[dr]_{\delta _1} && \Delta [1]\times \Delta [1] \ar[rr]^(.6){\Phi \circ (N\mu \times id)} && X \\
 & \Delta [2] \ar[ur]_{\gamma ^2_0} && \Delta [1] \ar[ll]^{\delta _2} \ar[ur]_{\Phi \varepsilon _j}
 }
\end{displaymath}
that establishes $\Phi \varepsilon _{j+1}$ as a face of the $2$-simplex
\[z_1=\Phi \circ (N\mu \times 1)\circ \gamma ^2_1\]
and $\Phi \varepsilon _j$ as a face of the $2$-simplex
\[z_0=\Phi \circ (N\mu \times 1)\circ \gamma ^2_0,\]
in such a way that $z_1\delta _1=z_0\delta _1$.

Recall that the $j$-th and the $(j+1)$-th vertex of the simplex $\Phi (\gamma ^n_{j+1})$ of $X$ are equal. This implies that
\[z_1=w_1\sigma _1.\]
Similarly, the $(j+1)$-st and the $(j+2)$-nd vertex of $\Phi (\gamma ^n_j)$ are equal, implying that $z_0=w_0\sigma _0$. It follows that $\Phi \varepsilon _j=\Phi \varepsilon _{j+1}$ as $\delta _1$ and $\delta _0$ are sections of $\sigma _0$ and $\delta _1$ and $\delta _2$ are sections of $\sigma _1$.
\end{proof}

\begin{lemma}\label{lem:part2_standard_1_simplex_as_exponent}
Let $\Phi$ be an $n$-simplex of $X^{\Delta [1]}$ such that the $k$-th vertex is equal to the $(k+1)$-th vertex, for some $k$ with $0\leq k<n$. Then there is an $(n-1)$-simplex $\Psi$ such that $\Phi =\Psi \sigma _k$.
\end{lemma}
\begin{proof}
For the purpose of constructing $\Psi$ we apply $N\sigma _k \times id$ to the elements of the preferred set $\{ \gamma ^{n+1}_0$, $\dots$, $\gamma ^{n+1}_n\}$ of generators of the prism. The result of the calculation is the set of equations
\begin{displaymath}
(N\sigma _k\times id)(\gamma ^{n+1}_j)=
\begin{cases}
\gamma ^n_j\sigma _{k+1}, & 0 \leq j \leq k \\ 
\gamma ^n_{j-1}\sigma _k, & k < j \leq n.
\end{cases}
\end{displaymath}
Should $\Psi$ exist, then it must therefore satisfy
\begin{displaymath}
\Phi (\gamma ^{n+1}_j)=
\begin{cases}
\Psi (\gamma ^n_j)\sigma _{k+1}, & 0 \leq j \leq k \\ 
\Psi (\gamma ^n_{j-1})\sigma _k, & k < j \leq n.
\end{cases}
\end{displaymath}
As $\delta _{k+1}$ is a section of both $\sigma _k$ and $\sigma _{k+1}$ we are lead to define a function
\[\psi :\{ \gamma ^n_0,\dots ,\gamma ^n_{n-1}\} \to X_n\]
by
\begin{displaymath}
\psi (\gamma ^n_j)=
\begin{cases}
\Phi (\gamma ^{n+1}_j)\delta _{k+1}, & 0\leq j\leq k \\ 
\Phi (\gamma ^{n+1}_{j+1})\delta _{k+1}, & k<j<n
\end{cases}
\end{displaymath}
that specifies where $\Psi$ sends the generators, if it exists.

Note the following regarding the definition of $\psi$. First, we have made the choices of the section $\delta _{k+1}$ of $\sigma _{k+1}$ and the section $\delta _{k+1}$ of $\sigma _k$. These choices seem to make the argument below as simple as possible. Second, we have that
\[\psi (\gamma ^n_k)=\Phi (\gamma ^{n+1}_k)\delta _{k+1}=\Phi (\gamma ^{n+1}_k\delta _{k+1})=\Phi (\gamma ^{n+1}_{k+1}\delta _{k+1})=\Phi (\gamma ^{n+1}_{k+1})\delta _{k+1}\]
due to (\ref{Equation_generator_compatibility}). This ensures that there is some compatibility between the two clauses of the definition of $\psi$ by cases. We take advantage of the equation below.

Crucially, the function $\psi$ obeys the compatibility criterion
\begin{equation}\label{equation_compatibility_def_psi}
 \psi (\gamma ^n_j)\delta _{j+1}=\psi (\gamma ^n_{j+1})\delta _{j+1}
\end{equation}
for $0\leq j<n-1$, as we now explain. There are three cases. Either $j<k$, $j=k$ or $j>k$.

First, we verify (\ref{equation_compatibility_def_psi}) in the case when $j=k$. For this we use (\ref{Equation_generator_compatibility}) and the general rule $\delta _i\delta _j=\delta _j\delta _{i-1}$ for $j<i$. We get that
\begin{displaymath}
\begin{array}{rcl}
\psi (\gamma ^n_k)\delta _{k+1} & = & (\Phi (\gamma ^{n+1}_k)\delta _{k+1})\delta _{k+1} \\
& = & (\Phi (\gamma ^{n+1}_{k+1})\delta _{k+1})\delta _{k+1} \\
& = & \Phi (\gamma ^{n+1}_{k+1})(\delta _{k+1}\delta _{k+1}) \\
& = & \Phi (\gamma ^{n+1}_{k+1})(\delta _{k+2}\delta _{k+1}) \\
& = & (\Phi (\gamma ^{n+1}_{k+1})\delta _{k+2})\delta _{k+1} \\
& = & (\Phi (\gamma ^{n+1}_{k+1}\delta _{k+2}))\delta _{k+1} \\
& = & (\Phi (\gamma ^{n+1}_{k+2}\delta _{k+2}))\delta _{k+1}
\end{array}
\end{displaymath}
and that
\begin{displaymath}
\begin{array}{rcl}
\psi (\gamma ^n_{k+1})\delta _{k+1} & = & (\Phi (\gamma ^{n+1}_{k+2})\delta _{k+1})\delta _{k+1} \\
& = & \Phi (\gamma ^{n+1}_{k+2})(\delta _{k+1}\delta _{k+1}) \\
& = & \Phi (\gamma ^{n+1}_{k+2})(\delta _{k+2}\delta _{k+1}),
\end{array}
\end{displaymath}
which confirms that (\ref{equation_compatibility_def_psi}) holds in the case when $j=k$.

Second, consider the case when $j<k$. We get that
\begin{displaymath}
\begin{array}{rcl}
\psi (\gamma ^n_j)\delta _{j+1} & = & (\Phi (\gamma ^{n+1}_j)\delta _{k+1})\delta _{j+1} \\
& = & \Phi (\gamma ^{n+1}_j)(\delta _{k+1}\delta _{j+1}) \\
& = & \Phi (\gamma ^{n+1}_j)(\delta _{j+1}\delta _k) \\
& = & (\Phi (\gamma ^{n+1}_j)\delta _{j+1})\delta _k \\
& = & (\Phi (\gamma ^{n+1}_j\delta _{j+1}))\delta _k \\
& = & (\Phi (\gamma ^{n+1}_{j+1}\delta _{j+1}))\delta _k
\end{array}
\end{displaymath}
and that
\begin{displaymath}
\begin{array}{rcl}
\psi (\gamma ^n_{j+1})\delta _{j+1} & = & (\Phi (\gamma ^{n+1}_{j+1})\delta _{k+1})\delta _{j+1} \\
& = & \Phi (\gamma ^{n+1}_{j+1})(\delta _{k+1} \delta _{j+1}) \\
& = & \Phi (\gamma ^{n+1}_{j+1})(\delta _{j+1}\delta _k), \\
\end{array}
\end{displaymath}
which confirms that (\ref{equation_compatibility_def_psi}) holds in the case when $j<k$.

Third, consider the case when $j>k$. We get that
\begin{displaymath}
\begin{array}{rcl}
\psi (\gamma ^n_j)\delta _{j+1} & = & (\Phi (\gamma ^{n+1}_{j+1})\delta _{k+1})\delta _{j+1} \\
& = & \Phi (\gamma ^{n+1}_{j+1})(\delta _{k+1}\delta _{j+1}) \\
& = & \Phi (\gamma ^{n+1}_{j+1})(\delta _{j+2}\delta _{k+1}) \\
& = & (\Phi (\gamma ^{n+1}_{j+1})\delta _{j+2})\delta _{k+1} \\
& = & (\Phi (\gamma ^{n+1}_{j+1}\delta _{j+2}))\delta _{k+1} \\
& = & (\Phi (\gamma ^{n+1}_{j+2}\delta _{j+2}))\delta _{k+1}
\end{array}
\end{displaymath}
and that
\begin{displaymath}
\begin{array}{rcl}
\psi (\gamma ^n_{j+1})\delta _{j+1} & = & (\Phi (\gamma ^{n+1}_{j+2})\delta _{k+1})\delta _{j+1} \\
& = & \Phi (\gamma ^{n+1}_{j+2})(\delta _{k+1} \delta _{j+1}) \\
& = & \Phi (\gamma ^{n+1}_{j+2})(\delta _{j+2}\delta _{k+1}). \\
\end{array}
\end{displaymath}
This confirms that (\ref{equation_compatibility_def_psi}) holds in the case when $j>k$ and concludes our verification of (\ref{equation_compatibility_def_psi}) for any $j$ with $0\leq j<n-1$.

We define $\Psi :\Delta [n-1]\times \Delta [1]\to X$ by letting
\[\Psi (\gamma ^n_j\alpha )=\psi (\gamma ^n_j)\alpha\]
for all $j$ with $0\leq j<n$. The map $\Psi$ is well defined and a simplicial map as $\psi$ satisfies the glueing condition (\ref{equation_compatibility_def_psi}). Thus it remains to argue that
\begin{equation}\label{eq:Phi_degenerate}
\Phi =\Psi \circ (N\sigma _k\times id).
\end{equation}
It suffices to check that the equation holds on the generators $\gamma ^{n+1}_0,\dots ,\gamma ^{n+1}_n$ for the prism $\Delta [n]\times \Delta [1]$.

We use the calculation of $(N\sigma _k\times id)(\gamma ^{n+1}_j)$, $0\leq j\leq n$, above. There are three cases. Either $0\leq j\leq k$, $j=k+1$ or $j>k+1$.

If $0\leq j\leq k$, then
\begin{displaymath}
\begin{array}{rcl}
\Psi \circ (N\sigma _k\times id)(\gamma ^{n+1}_j) & = & \Psi (\gamma ^n_j\sigma _{k+1}) \\
& = & \psi (\gamma ^n_j)\sigma _{k+1} \\
& = & (\Phi (\gamma ^{n+1}_j)\delta _{k+1})\sigma _{k+1} \\
& = & \Phi (\gamma ^{n+1}_j),
\end{array}
\end{displaymath}
which confirms (\ref{eq:Phi_degenerate}) for the generators $\gamma ^{n+1}_0, \dots \gamma ^{n+1}_k$. This is because the vertices of $\Phi (\gamma ^{n+1}_j)$ that are numbered $k+1$ and $k+2$ are equal. Thus the simplex splits off $\sigma _{k+1}$ by \cref{lem:degenerate_part_factorization_through} as $X$ is non-singular. Furthermore, $\delta _{k+1}$ is a section of $\sigma _{k+1}$.

Note that $\Phi (\gamma ^{n+1}_j)$ splits off $\sigma _k$ when $j>k$. This is because the vertices of $\Phi (\gamma ^{n+1}_j)$ that are numbered $k$ and $k+1$ are equal. Thus the simplex splits off $\sigma _k$ by \cref{lem:degenerate_part_factorization_through} as $X$ is non-singular. Furthermore, $\delta _{k+1}$ is a section of $\sigma _k$.

Consider the case when $j=k+1$. We get that
\begin{displaymath}
\begin{array}{rcl}
\Psi \circ (N\sigma _k\times id)(\gamma ^{n+1}_{k+1}) & = & \Psi (\gamma ^n_k\sigma _k) \\
& = & \psi (\gamma ^n_k)\sigma _k \\
& = & (\Phi (\gamma ^{n+1}_k)\delta _{k+1})\sigma _k \\
& = & (\Phi (\gamma ^{n+1}_{k+1})\delta _{k+1})\sigma _k \\
& = & \Phi (\gamma ^{n+1}_{k+1}),
\end{array}
\end{displaymath}
which confirms (\ref{eq:Phi_degenerate}) for the generator $\gamma ^{n+1}_{k+1}$.

Finally, we consider the case when $j>k+1$. Then
\begin{displaymath}
\begin{array}{rcl}
\Psi \circ (N\sigma _k\times id)(\gamma ^{n+1}_j) & = & \Psi (\gamma ^n_{j-1}\sigma _k) \\
& = & \psi (\gamma ^n_{j-1})\sigma _k \\
& = & (\Phi (\gamma ^{n+1}_j)\delta _{k+1})\sigma _k \\
& = & \Phi (\gamma ^{n+1}_j),
\end{array}
\end{displaymath}
which confirms (\ref{eq:Phi_degenerate}) for the generators $\gamma ^{n+1}_{k+2}, \dots ,\gamma ^{n+1}_n$. This concludes our verification of (\ref{eq:Phi_degenerate}). Thus $\Phi$ is a degenerate simplex of $X^{\Delta [1]}$.
\end{proof}




\bibliographystyle{unsrt}  
\bibliography{main}

\end{document}
