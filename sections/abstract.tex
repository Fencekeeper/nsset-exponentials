
\begin{abstract}
\noindent A simplicial set is \textbf{non-singular} if the representing maps of its non-degenerate simplices are degreewise injective. The category of simplicial sets has a \textbf{simplicial mapping set} $X^K$ whose set of $n$-simplices are the simplicial maps $\Delta [n]\times K\to X$. We prove that $X^K$ is non-singular whenever $X$ is non-singular.

\vspace{1em}
MSC-class: 18D15 (Primary), 55U10 (Secondary)
\end{abstract}